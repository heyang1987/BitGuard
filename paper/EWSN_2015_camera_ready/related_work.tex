\vspace{-15pt}
\section{Related Work}\label{sec:related_work}
\vspace{-10pt}
One of the primary hardware-level radiation hardening approaches is Silicon-on-Insulator (SoI) technology, used in microprocessor fabrication~\cite{Celler2003SOI}. The design improves the circuit's tolerance to highly-charged particles, reducing the chance of SEU occurrence. Irom et al.~\cite{Irom2002SOI} compare SEU error rates in  SoI microprocessors to conventional microprocessors. SEU rates were observably lower in SoI microprocessors. Though SoI technology protects systems from SEUs, it prevents developers from using commercial off-the-shelf (COTS) devices, increasing system cost due to the high price of SoI circuits.

Redundancy is a widely used fault-tolerance technique, both via hardware and software. The Triple Modular Redundancy (TMR)~\cite{TMR} approach executes instructions on three unique systems. A voting module is used to compare the results and choose the common result. Due to the low probability that more than one SEU will occur simultaneously at the same geographic location in more than one device~\cite{underwood1992observations}, TMR is a popular SEU protection technique and allows the use of COTS components. However, hardware-based TMR introduces significant hardware overhead and power consumption, which can present concerns for weight-limited and power-critical systems.

Time Redundancy~\cite{ulta2013} is a software-only redundancy technique which runs each instruction three times on a single processor. The results are stored, and a voting module is invoked to yield the (most) common result. Error Detection by Duplicated Instructions (EDDI)~\cite{oh2002error}, a variation on Time Redundancy, duplicates each instruction during the compilation phase and assigns each different registers and memory space. As a result, EDDI is able to protect systems from not only data SEUs, but also instruction SEUs. Time Triple Modular Redundancy~\cite{ulta2013} is a combination of time redundancy and hardware-based TMR. Each instruction is executed by three unique systems, as in standard TMR, but the systems execute the instruction in different clock cycles in a time-redundant fashion. This allows more instructions to be executed in parallel.

A watchdog timer (WDT)~\cite{huang1986watchdog} is a timer used to detect and recover from system crashes by repeatedly querying the protected system and resetting the system if no response is received. A software-based WDT is straightforward to implement and introduces little overhead. However, it suffers the risk that an SEU may cause the WDT itself to malfunction. Despite increased cost, hardware-based WDT provides a reliable solution. Note, however, that WDT is typically used with other techniques since it only detects severe system faults.

Shirvani et al.~\cite{Shirvani2001EDAC} examine a set of Error Detection and Correction (EDAC) methods used to detect and correct errors in memory, such as those caused by SEUs. The authors find that the reliability of software-based methods tends to decrease over time more rapidly than hardware-based methods. However, the loss rate is low enough that software-based methods are still more cost effective than hardware-based methods.  
