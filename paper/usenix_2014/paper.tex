% TEMPLATE for Usenix papers, specifically to meet requirements of
%  USENIX '05
% originally a template for producing IEEE-format articles using LaTeX.
%   written by Matthew Ward, CS Department, Worcester Polytechnic Institute.
% adapted by David Beazley for his excellent SWIG paper in Proceedings,
%   Tcl 96
% turned into a smartass generic template by De Clarke, with thanks to
%   both the above pioneers
% use at your own risk.  Complaints to /dev/null.
% make it two column with no page numbering, default is 10 point

% Munged by Fred Douglis <douglis@research.att.com> 10/97 to separate
% the .sty file from the LaTeX source template, so that people can
% more easily include the .sty file into an existing document.  Also
% changed to more closely follow the style guidelines as represented
% by the Word sample file. 

% Note that since 2010, USENIX does not require endnotes. If you want
% foot of page notes, don't include the endnotes package in the 
% usepackage command, below.

% This version uses the latex2e styles, not the very ancient 2.09 stuff.
\documentclass[letterpaper,twocolumn,10pt]{article}

\usepackage{usenix,epsfig,endnotes}
\usepackage[utf8]{inputenc}
\usepackage{cite}
\usepackage{url}
\usepackage{comment}
\usepackage{listings}
\usepackage{multirow}
\usepackage{hyperref}
\usepackage{graphicx}
%\usepackage{subfig}
%\usepackage{tocloft}
%\usepackage{subfigure}
\usepackage{subcaption}
\usepackage{tabularx}
\usepackage{color}
\usepackage{multicol}
\usepackage{float}
\usepackage{amsmath}

\urlstyle{same}

\lstset{frame=tb,
  language=C,
  aboveskip=3mm,
  belowskip=3mm,
  frame=single,
  showtabs=false,
  showstringspaces=false,
  columns=flexible,
  basicstyle={\small\ttfamily},
  numbers=left,
  numberstyle=\small,
  numbersep=5pt,
  keywordstyle=\color{blue},
  morekeywords={bool, uint8_t, uint16_t, int8_t},
  commentstyle=\color{green},
  stringstyle=\color{mauve},
  breaklines=true,
  breakatwhitespace=true,
  tabsize=3,
  xleftmargin=12pt,
  xrightmargin=5pt,
  captionpos=b
}
\lstset{emph={%  
    Configuration, Target, Platform, MCU, Interface, Version, 
    Global, ResponseTimeout, InterByteTimeout, Retry,
    Uart, BaudRate, Data, Stop, Parity, 
    Function, Name, EntryPoint, Command, ResponseMatch, Delay, Dependency, DName
    },emphstyle={\color{red}\textbf}%
}%

\begin{document}

%don't want date printed
\date{}

%make title bold and 14 pt font (Latex default is non-bold, 16 pt)
\title{\Large \bf A Software Approach to Protecting Embedded System Memory\\ from Single Event Upsets}

%for single author (just remove % characters)
\author{
{\rm Jiannan Zhai, Yangyang He, Fred S. Switzer, Jason O. Hallstrom}\\
School of Computing, Clemson University\\
\{jzhai, yyhe, skyler, jasonoh\}@clemson.edu
% copy the following lines to add more authors
% \and
% {\rm Name}\\
%Name Institution
} % end author

\maketitle
\thispagestyle{empty}
\begin{abstract}
Radiation from radioactive environments, such as those encountered during space flight, can cause damage to embedded systems. One of the most common examples is the \textit{single event upset} (SEU), which occurs when a high-energy ionizing particle passes through an integrated circuit, changing the value of a single bit by releasing its charge. The SEU could cause damage and potentially fatal failures to spacecraft and satellites. In this paper, we present an approach that extends the AVR-GCC compiler to protect the system stack from SEUs through duplication, validation, and recovery. 
%save Our approach injects assembly code into the target application to achieve memory protection without introducing additional hardware. 
Three applications are used to verify our approach, and the time and space overhead characteristics are evaluated.
\end{abstract}


% A category with the (minimum) three required fields
%\category{H.4}{Information Systems Applications}{Miscellaneous}
%A category including the fourth, optional field follows...
%\category{D.2.8}{Software Engineering}{Metrics}[complexity measures, performance measures]

%\terms{Theory}


\vspace{-35pt}
\section{Introduction}\label{sec:introduction}
\vspace{-10pt}
%save Humans have a longstanding curiosity about outer space. Since the launch of the first artificial satellite, Sputnik 1~\cite{sputnik1}, in 1957, over 6,000 satellites have been launched into space. 
There are more than 1,000 operating satellites in orbit around Earth~\cite{satellite:total} today, and 
%save an estimated 1,200 satellites will be launched over the next decade~\cite{satellite:next10years}.
as of 2012, more than 130 manned spacecraft have been launched by the United States alone~\cite{space:shuttle:list}. 
%save There are currently two operational space stations, and seven more are planned over the next decade~\cite{space:station:tiangong2}\cite{space:station:almaz}\cite{space:station:opsek}\cite{space:station:tiangong3}. Given the high cost and vital importance of spacecraft rovers and satellites, as well as their increasing functionality and complexity, the hardware and software reliability requirements are stringent.
One of the most important factors that affect the reliability of such systems 
%save spacecraft 
(and other equipment) is the 
%save quality 
reliability of the constituent embedded components which control the underlying telemetry systems, command systems, attitude control systems, and more~\cite{fundamentals:space}. 
%save For example, the MSX (Midcourse Space Experiment) spacecraft, launched in the mid-1980s, was equipped with 54 embedded processors, running more than 275,000 lines of code, managing 19 subsystems~\cite{fundamentals:space}. 
Embedded software failures can cause serious consequences in this context. 
%save In 1996, the Ariane 5 spacecraft, which took 10 years and 7 billion (US) dollars to build, crashed due to the failure of the Flight Control Subsystem~\cite{ariane5}.

The environment outside the Earth's atmosphere is highly radioactive 
%save The radiation is generated mainly by the sun and other stars 
and can cause damage to semiconductor devices~\cite{fundamentals:space}. One of the most common types of damage caused by radiation is the \textit{single event upset} (SEU). Extremely small electronic components (i.e., tens of nanometers~\cite{intel:chip:size}) are used in modern integrated circuitry; the components cannot carry much charge. As a result, one high-energy ionizing particle passing through an integrated circuit can release enough charge to change the state of a binary digit, causing a stored bit to change to its opposite value (i.e., a 0-bit can become a 1-bit, and vice-versa~\cite{fundamentals:space}). The
%save damage caused by an SEU 
result can range from system malfunction to system crash.

Modern approaches used to prevent and correct SEU errors often introduce additional hardware to the target system. In this paper, we present a {\em software-only} approach that detects and corrects SEUs in RAM. The paper focuses on the system stack, which is the most important and dynamic region in memory. The system stack is protected by injecting code into the target assembly generated by AVR-GCC. 
%save After injection, each callee computes and saves the checksum of its caller's current stack frame and duplicates the caller's stack frame when the callee enters its function body. Before the callee returns, it verifies the stack frame of the caller using the saved checksum and overwrites the stack frame using the duplicate, if an SEU is detected. 
Our approach 
%save changes the target system software and 
does not introduce additional hardware, and since it operates at the assembly level, it is language and application neutral. 
%save To demonstrate our approach, an AVR microprocessor, the ATmega644~\cite{atmel:avr}, is used. 

The main contributions of our work are as follows: (i) We present an approach that protects the system stack by injecting assembly code at the beginning and end of each application function. (ii) We present an implementation of the approach, using the popular AVR architecture as a target. (iii) We verify the protection efficacy of our approach and evaluate performance in terms of space and speed overhead using three applications with different stack usage patterns.

%save {\bf Paper Organization.} Section \ref{sec:related_work} summarizes key elements of related work. Section \ref{sec:background} provides background related to our approach, including the microprocessor architecture, AVR function call process, and AVR toolchain. Section \ref{sec:design} presents the design and implementation of our approach. Section \ref{sec:evaluation} presents an evaluation of the approach, with an emphasis on ROM size and execution speed overhead. Finally, Section \ref{sec:conclusion} concludes with a summary of contributions and pointers to future work.
\section{Related Work}\label{sec:related_work}
Single event upset mitigation has been explored from both hardware and software perspectives. 
%save Common hardware solutions focus on hardware design modifications to prevent radiation from causing SEUs, and extensions to correct SEUs. Common software solutions adopt one or more error-correcting codes to detect and correct SEUs. In practice, hardware and software are usually both used to achieve better SEU protection, and to balance among cost, execution efficiency, and power consumption.
One of the primary hardware-level radiation hardening approaches is Silicon-on-Insulator (SoI) technology, used in microprocessor fabrication~\cite{Celler2003SOI}. 
%save In this approach, transistors are placed on a thin layer of silicon, which is then placed on top of an insulator. 
The design improves the circuit's tolerance to highly-charged particles, reducing the chance of SEU occurrence. 
%Moreover, SoI technology reduces the capacitance of the switches and the size of the processors, and thus reduces the area over which highly-charged particles can strike, statistically reducing the likelihood of impacts, and therefore errors.
Irom et al.~\cite{Irom2002SOI} compare SEU error rates in  SoI microprocessors to conventional microprocessors. They subject both types of microprocessors to proton and heavy-ion impacts, both known to cause SEUs. SEU rates were observably lower than those in commercial microprocessors. Though SoI technology protects systems from SEUs, it prevents developers from using commercial off-the-shelf (COTS) devices, increasing system cost due to the high price of SoI circuits.

Redundancy is a widely used fault-tolerance technique, both via hardware and software. The Triple Modular Redundancy (TMR)~\cite{TMR} approach executes instructions on three unique systems. A voting module is used to compare the results and choose the common result. Due to the low probability that more than one SEU will occur simultaneously at the same geographic location in more than one device~\cite{underwood1992observations}, TMR is a popular SEU protection technique and allows the use of COTS components. However, hardware-based TMR introduces significant hardware overhead and power consumption, which can present concerns for weight-limited and power-critical systems.

Time Redundancy~\cite{ulta2013} is a software-only redundancy technique which runs each instruction three times on a single processor. The results are stored, and a voting module is invoked to yield the (most) common result. Error Detection by Duplicated Instructions (EDDI)~\cite{oh2002error}, a variation on Time Redundancy, duplicates each instruction during the compilation phase and assigns each different registers and memory space. As a result, EDDI is able to protect systems from not only data SEUs, but also instruction SEUs. Time Triple Modular Redundancy~\cite{ulta2013} is a combination of time redundancy and hardware-based TMR. Each instruction is executed by three unique systems, as in standard TMR, but the systems execute the instruction in different clock cycles in a time-redundant fashion. This allows more instructions to be executed in parallel.
%save Others have worked on replacing memory cells by SEU-hardened cells and circuits. \cite{weaver1987seu} presents LRAM, which adds decoupling resistors to slow down long pulses, enabling cells to differentiate pulses generated by an SEU and a write signal. She et al.~\cite{She2012Latch} improve the design of conventional latches by implementing an error detection circuit and integrated multiplexer. While conventional latches are susceptible to voltage changes caused by SEUs, the proposed latch uses an error detection circuit that checks for faults using the precharge and discharge operations. The latch uses a multiplexer to output a corrected signal based on the fault detected by the error detection circuit. The authors found that the proposed latch introduces little overhead and offers good performance, as well as better SEU protection than conventional latches.

A watchdog timer (WDT)~\cite{huang1986watchdog} is a timer used to detect and recover from system crashes by repeatedly querying the protected system and resetting the system if no response is received. A software-based WDT is straightforward to implement and introduces little overhead. However, 
%save Since the timer runs along with the protected system on the same device, 
it suffers the risk that an SEU may cause the WDT itself to malfunction. Despite increased cost, hardware-based WDT provides a reliable solution. Note, however, that WDT is typically used with other techniques since it only detects severe system faults.

Shirvani et al.~\cite{Shirvani2001EDAC} examine a set of Error Detection and Correction (EDAC) methods used to detect and correct errors in memory, such as those caused by SEUs. 
%save EDAC methods come in both hardware and software forms. In scenarios where hardware-based approaches are cost prohibitive, software-based methods work well. 
Four software-based coding schemes are considered, comprising Hamming, Cyclic, Parity, and Reed-Solomon codes. The authors find that the reliability of software-based methods tends to decrease over time more rapidly than hardware-based methods. However, the loss rate is low enough that software-based methods are still more cost effective than hardware-based methods.  
%save The authors found that most EDAC implementations can be improved by periodically scrubbing (completely cleaning) memory.
%save Mhatre and Aras~\cite{mhatreSeuTmr} present a design for the on-board computer of the COEP Student Satellite, a HAM communication pico-satellite. SEU protection on the satellite involves the implementation of Hamming codes, Triple Modular Redundancy, and watch dog software. In the Hamming code, each 32-bit instruction is coded in the form of a 38-bit codeword, where the redundant bits are used for parity. Single bit errors are corrected by comparing the parity values with pre-calculated values. Triple Modular Redundancy is used in storing and protecting these parity values, saving space by not implementing TMR over the entire instruction memory. This extra storage increases program memory requirements by 75\%, rather than the 200\% increase required by a full implementation of TMR. The watch dog software prevents other errors by automatically escaping system crashes.

Dutton and Stroud present a design implemented in configurable logic blocks for SEU detection and correction in the configuration memory of Field Programmable Gate Arrays (FPGAs)~\cite{CATA09seuonVirtex}. The architecture of the Xilinx Virtex FPGA is modified to implement an SEU controller that uses Hamming codes and parity values to detect and correct single bit errors in memory. This combination of Hamming and parity can also detect multiple bit upsets, but correction is not possible. The benefits include protection of the controller from SEUs and the high speed of error detection and correction, as compared to other methods.




\begin{comment}

Prior work on single event upset mitigation spans two categories:
First, robustness at the device level may be improved by building the hardware in such a way that radiation does not cause upset events. 
Common hardware modifications include physical shields, protective insulation, and error correcting latches.
Second, design level robustness may be improved by incorporating libraries and software designed to protect from single event upsets through the use of Hamming or other codes, or triple modular redundancy.

\subsection{Device-Level Robustness}

One of the primary methods of device-level radiation hardening is to fabricate processors using Silicon on Insulator (SOI) technology.
In this process, transistors are placed on a thin layer of silicon, which is then placed on top of an insulator, reducing the capacitance of the switches and the size of processors~\cite{Celler2003SOI}.
Reducing processor size effectively reduces the area over which highly-charged particles can strike, statistically reducing the likelihood of impacts, and therefore errors.

Irom et al.~\cite{Irom2002SOI} compare SEU error rates in  SOI microprocessors to conventional microprocessors.
They subject both types of microprocessors to proton impacts within a cyclotron, and to heavy-ion impacts within an accelerator, both of which are known to cause SEUs in processors.
From these tests, Irom et al. conclude that due to the significant reduction of cross sections in SOI microprocessors, SEU rates are lower than those in commercial microprocessors.

She et al.~\cite{She2012Latch} improve the design of conventional latches by implementing an error detection circuit and integrated multiplexer.
While conventional latches are susceptible to voltage changes caused by SEUs, the proposed latch uses an error detection circuit that checks for faults using the precharge and discharge operations.
The latch uses a multiplexer to output a corrected signal based on the fault detected by the error detection circuit.
The authors found that the proposed latch introduces little overhead and offers good performance, as well as better SEU protection than conventional latches.

\subsection{Design-Level Robustness}

The most popular methods of radiation hardening are at the design level, foregoing the need to modify the hardware, thereby allowing the use of commercial microcontrollers.
Some of these methods involve parity bits and linear error-correcting codes~\cite{ErrorCorrectingCodes}, such as the Hamming code, which correct bit errors by storing extra information about data in larger blocks.
Alternately, Triple Modular Redundancy (TMR)~\cite{TMR} is a voting-based approach that prevents errors by implementing three systems, and then using the common result.
The motivation is that the likelihood of two single event upsets occurring at precisely the same time, in the same geographic location, in two different devices, is incredibly low, making TMR popular in spite of the added expense.
While this method can be implemented via hardware (such as in RAID~\cite{RAID}), it is also effective as a software-only strategy.

Shirvani et al.~\cite{Shirvani2001EDAC} examine a set of error detection and correction (EDAC) methods.
These methods detect and subsequently correct errors in memory, such as those caused by SEUs.
EDAC methods come in both hardware and software forms.
In scenarios where hardware-based approaches are cost prohibitive, software-based methods work well.
The authors found that the reliability of software-based methods tends to decrease over time more rapidly than hardware-based methods.
However, the rate of reliability loss is low enough to still be more cost effective than hardware-based methods.
% but at a rate that it would still be cost-efficient to forgo hardware hardening in favor of software-based error correction.
Four software-based coding schemes were considered, comprising Hamming, Cyclic, Parity, and Reed-Solomon codes.
The authors found that most EDAC implementations can be improved by periodically scrubbing (completely cleaning) memory.
%EDAC is simply a blanket term for most of the error detections and error recovery methods outlined in this article, and this Stanford tech report goes into detail on many of them, providing useful statistical analysis.

Mhatre and Aras~\cite{mhatreSeuTmr} present a design for the on-board computer of the COEP Student Satellite, a HAM communication pico-satellite. 
SEU protection on the satellite involves the implementation of Hamming codes, Triple Modular Redundancy, and watch-dog software.
In the Hamming code, each 32-bit instruction is coded in the form of a 38-bit codeword, where the redundant bits are used for parity.
Single bit errors are corrected by comparing the parity values with pre-calculated values.
Triple Modular Redundancy is used in storing and protecting these parity values, saving space by not implementing TMR over the entire instruction memory.
This extra storage increases program memory requirements by 75\%, rather than the 200\% increase required by a full implementation of TMR.
The watch-dog software prevents other errors by automatically escaping system crashes.


Similarly, Dutton and Stroud present a design implemented in configurable logic blocks for SEU detection and correction in the configuration memory of field programmable gate arrays (FPGAs)~\cite{CATA09seuonVirtex}.
The architecture of the Xilinx Virtex FPGA is modified to implement an SEU controller that uses Hamming codes and parity values to detect and correct single bit errors in memory.
This combination of Hamming and parity can also detect multiple bit upsets, but correction is still not possible.
The benefits include the protection of the controller from SEUs and the high speed of error detection and correction, as compared to other methods.

\end{comment}
\vspace{-20pt}
\section{Background}\label{sec:background}
\vspace{-10pt}
While our approach is architecture neutral, our implementation is based on the Atmel AVR toolchain and focuses on AVR microprocessors. In this section, we survey the stack frame, function call process, and the AVR Toolchain optimization levels.
\vspace{-10pt}
\subsection{Stack Frame}
\vspace{-10pt}
The stack consists of stack frames, each corresponding to a function call. A stack frame is created when a function is called, and freed when the function returns. For example, as shown in Figure \ref{fig:stack_frame}, when the \textit{main} function calls the \textit{foo} function, a stack frame will be created for \textit{foo}. First, the return address of \textit{main} will be pushed to the stack, followed by the conflict registers. Next, the local variables and parameters will be pushed to the stack in reverse order of declaration. The stack frame spans the return address through the first parameter. The stack frame pointer, Y, now points to the next available address in the stack. When \textit{foo} finishes execution, the stack frame will be freed, and the stack frame pointer will point back to the position where the return address of the previous stack frame was stored.
\vspace{-15pt}
\begin{figure}
\centering
\includegraphics[scale=0.55]{figures/stack_frame_v2.pdf}
\vspace{5pt}
\caption{AVR Stack Frame}
\label{fig:stack_frame}
\end{figure}
\vspace{-20pt}
\vspace{-10pt}
\subsection{Function Calls}
\vspace{-10pt}
All function calls follow the same process and use the system stack to perform most operations, as illustrated in Figure \ref{fig:original_function_operation}. Figure \ref{fig:original_function_operation_process} explains the execution process when a function is called, and Figure \ref{fig:original_function_operation_stack} shows the associated stack changes after each operation is performed. Each rectangle represents two bytes in the stack. The numbers below each stack denote the operation(s) that changed the stack. \texttt{SP} denotes the stack pointer, and \texttt{Y} denotes the stack frame pointer. When a function is called, the return address is automatically pushed onto the stack by one of the function call instructions, \texttt{call}, \texttt{rcall}, or \texttt{icall} (step 1). After the stack frame pointer is pushed (step 2), the stack frame of the function is created by changing the stack pointer and stack frame pointer (step 3). The arguments and local variables are then pushed onto the stack (step 4), and the function begins executing (step 5). The arguments and local variables are released after the function finishes its execution (step 6), and the stack frame pointer is restored (step 7). Finally, the function returns (step 8). The return address is popped and used when one of the function return instructions, \texttt{ret} or \texttt{reti}, is called.
\begin{figure}
	\centering
	\begin{subfigure}[b]{0.4\columnwidth}
		\includegraphics[width=\textwidth, height=12cm]{figures/original_function_operations_process_v3}
		\caption{Process}
		\label{fig:original_function_operation_process}
	\end{subfigure}~
	\begin{subfigure}[b]{0.6\columnwidth}
		\includegraphics[width=\textwidth, height=11.5cm]{figures/original_function_operations_stack_v2}
		\caption{Stack}
		\label{fig:original_function_operation_stack}
	\end{subfigure}
	%	\vspace{5pt}
	\caption{Function Execution}\label{fig:original_function_operation}
\end{figure}
\vspace{-15pt}
\vspace{-10pt}
\subsection{AVR Toolchain Optimization Levels}
\vspace{-10pt}
The AVR GCC toolchain is used in our approach. It provides 5 optimization levels, each providing different optimization options. The exception is -O0, which offers no optimization\cite{hoste2008cole}. Our approach is based on modifying unoptimized assembly code generated with the -O0 option. This option makes it more convenient for developing, debugging, and evaluating our approach. However, we plan to extend our approach to other optimization levels in future work.


\begin{comment}
\subsection{AVR Architecture}

AVR microprocessors are based on a modified Harvard architecture~\cite{argade1996apparatus}, which stores instructions and data in physically separate memories, flash memory and SRAM, respectively. Instructions and data are accessed concurrently through separate memory buses. Flash memory is non-volatile and offers high capacity, but slow access speed, and is used to store executable programs composed of AVR instructions. The SRAM is volatile and offers low capacity, but fast access speed, and is used to store data used by the executable programs at runtime. The ATmega644 includes 64KB of flash memory, 4KB of SRAM, a 16-bit instruction bus, and an 8-bit data bus.

\subsubsection{AVR SRAM}

The on-board SRAM of the ATmega644 has an address range of 0x0100 to 0x10FF, as shown in Figure \ref{fig:ram_map}. The SRAM is partitioned into sections, each used to store different types of data. The \textit{.data} section is used to store initialized static variables and global variables. The \textit{.bss} section is used to store uninitialized static and global variables. The pre-allocated SRAM usage is the sum of the sizes of the .data and .bss sections. The remaining space in SRAM is shared by the heap and stack sections. The \textit{heap} section is used to store dynamically allocated memory, e.g. when \textit{malloc()} is called~\cite{goldt1995linux}. The heap grows ``upward'', towards the higher address range. The \textit{stack} section is used to store a return address, actual parameters, conflict registers, local variables, and other information. The stack grows ``downward'', from \textit{RAM\_END}, address 0x10FF, towards the lower address range.
\begin{figure}
\centering
\includegraphics[width=0.5\textwidth]{figures/Memory_model.pdf}
\caption{AVR RAM Map}
\label{fig:ram_map}
\end{figure}

\subsubsection{Stack Frame}

The stack consists of stack frames, each corresponding to a function call. Each stack frame is created when a function is called, and freed when the function returns. For example, as shown in Figure \ref{fig:stack_frame}, when the \textit{main} function calls the \textit{foo} function, a stack frame will be created for \textit{foo}. First, the return address of \textit{main} will be pushed to the stack, followed by the conflict registers. Next, the local variables and parameters will be pushed to the stack in reverse order of their declarations. The stack frame spans the return address through the first parameter. The stack frame pointer, Y, now points to the next available address in the stack. When \textit{foo} finishes execution, the stack frame will be freed, and the stack frame pointer will point back to the position where the return address of the previous stack frame was stored.

\begin{figure}
\centering
\includegraphics[scale=0.55]{figures/stack_frame_v2.pdf}
\caption{AVR Stack Frame}
\label{fig:stack_frame}
\end{figure}

\subsubsection{Registers}

AVR microprocessors have two types of registers, general-purpose registers and I/O registers. 

General-purpose registers are used for arithmetic operations, such as adding, subtracting, and comparing numbers, as well as indexing and setting long jump destinations. The ATmega644 has 32 general-purpose registers, R1 through R32, which are mapped into the first 32 locations of the RAM space, and can be directly used in assembly commands. Some general-purpose registers are used for special purposes; for example, R29 and R28 store a 16-bit address, the Y pointer, to indicate the top of the current stack frame. (The use of the Y pointer will be explained in the next subsection). The use of these registers is compiler-dependent. For example, AVR-GCC uses R24 and R25 to store the return value of each function call. We attempt to limit the number of registers manipulated by our approach to reduce the cost of saving and restoring the conflict registers.

The I/O registers are used to control the internal peripherals of the AVR microprocessor.
The ATmega644 has 64 I/O registers, mapped into the next 64 locations of the SRAM space, \textit{0x20} through \textit{0x5F}. Again, some I/O registers are used for special purposes; for example, AVR-GCC uses \textit{0x3E} and \textit{0x3D} as the stack pointer (SP), which indicates the current top of the stack.

\subsection{Function Calls}

All function calls follow the same process and use the system stack to perform most operations, as illustrated in Figure \ref{fig:original_function_operation}. Figure \ref{fig:original_function_operation_process} explains the execution process when a function is called, and Figure \ref{fig:original_function_operation_stack} shows the associated stack changes after each operation is performed. Each rectangle represents two bytes in the stack. The numbers below each stack denote the operation(s) that changed the stack. \texttt{SP} denotes the stack pointer, and \texttt{Y} denotes the stack frame pointer. When a function is called, the return address is automatically pushed onto the stack by one of the function call instructions, \texttt{call}, \texttt{rcall}, or \texttt{icall} (step 1). After the stack frame pointer is pushed (step 2), the stack frame of the function is created by changing the stack pointer and stack frame pointer (step 3). The arguments and local variables are then pushed onto the stack (step 4), and the function begins executing (step 5). The arguments and local variables are released after the function finishes its execution (step 6), and the stack frame pointer is restored (step 7). Finally, the function returns (step 8). The return address is popped and used when one of the function return instructions, \texttt{ret} or \texttt{reti}, is called.

\begin{figure}[h]
        \centering
        \begin{subfigure}[b]{0.4\columnwidth}
                \includegraphics[width=\textwidth, height=12cm]{figures/original_function_operations_process_v3}
                \caption{Process}
                \label{fig:original_function_operation_process}
        \end{subfigure}~
        \begin{subfigure}[b]{0.6\columnwidth}
                \includegraphics[width=\textwidth, height=11.5cm]{figures/original_function_operations_stack_v2}
                \caption{Stack}
                \label{fig:original_function_operation_stack}
        \end{subfigure}
        \caption{Function Execution}\label{fig:original_function_operation}
\end{figure}

\subsection{Atmel AVR Toolchain}

The Atmel AVR toolchain is a collection of tools used to generate executable programs for AVR microprocessors. The toolchain consists of the following tools.

\begin{itemize}

\item \textit{avr-gcc}, an extension of GNU GCC, is a cross compiler, which translates high-level C or C++ code to assembly code for AVR microprocessors.

\item \textit{avr-as} is an assembler, which translates AVR assembly code to an object file.

\item \textit{avr-ld} is a linker, which uses a linker script to combine object modules into an executable image suitable for loading into the flash memory of an AVR microprocessor. By using a customized linker script, the default locations and sizes in SRAM can be changed. New memory sections may also be added~\cite{sram}.

\item \textit{avr-libc} is a standard C library, which contains standard C routines, as well as additional AVR-specific library functions.

\end{itemize}

As a matter of convenience, the AVR toolchain can be used to compile, assemble, and link C programs in a single command. However, in order to modify the assembly code of an AVR application and use a customized linker script, these steps are performed individually.


AVR GCC provides 5 optimization levels, -O0, -O1, -O2, -O3, and -Os, each providing different optimization options. The exception is -O0, which offers no optimization\cite{hoste2008cole}. Our approach is based on modifying unoptimized assembly code generated with the -O0 option. This option makes it more convenient for developing, debugging, and evaluating our approach. However, we plan to extend our approach to other optimization levels in future work.

\end{comment}
\section{System Design/Implementation}\label{sec:design}

Since our approach focuses on the stack in RAM, we make the following assumptions. i) The flash memory and registers are not affected by SEUs. ii) Two or more flipped bits cannot concurrently exist in RAM. iii) The stack frame of the current function is not affected by SEUs. The reason for the assumptions is that in a single-processor system, the detection and correction of a given memory region depend on system registers and other memory regions, such as stack, .data section, etc. If they are affected by SEUs, the detection and correction process cannot work correctly. Moreover, it is rare that two or more bits get flipped at the same time, as mentioned in [find something to support this].

Our approach protects the system stack from being affected by SEUs, by injecting assembly code into the assembly code generated from the target C source code. The code is injected at both the beginning and the end of each function, and handles CRC calculation, memory duplication, etc. When a function is called, the code injected in the beginning of the callee calculates the CRC of the caller's stack frame and saves the CRC and caller's stack frame. Before the callee returns, the code injected in the end calculates the CRC of the caller's stack frame again, compares it with the saved CRC, and restores the caller's stack frame if the two CRCs do not match.

The \texttt{ASM Handler}, a tool written in Java, is created to handle the code inject, shown in Figure \ref{fig:code_inject_process}. First, the target C source code is compiled to assembly code by GCC. Again, the optimization level is set to \texttt{none}. Then, the ASM Handler scans the assembly code and injects customized assembly code into it based on the state machine. Finally, the modified assembly code is assembled and linked into the AVR executable code. In this Section, we discuss the supporting memory sections, the code segments to be injected to the target program, architecture of the ASM Handler, and the function execution process after code injection.

\subsection{Supporting Memory sections}\label{sec:memory_sections}

To store the stack frame duplicates, two new sections are created in SRAM after the \texttt{.bss} section by modifying the linker script~\cite{linkerscript}, shown in Figure xxx. 

The \texttt{md} section is used to store the stack frame duplicates, and is design to be a LIFO (Last-In-First-Out) stack structure, called SFS (\texttt{Stack Frame Snapshots}). The heap section grows towards the stack, and the usage of the heap and stack during runtime is unpredictable. To avoid the \texttt{md} and \texttt{heap} sections overlap each other, the size of the \texttt{md} section is fixed and is determined by the factor that indicates whether the heap is used in the target program, discussed in Section xxx. If the heap is used, the size of the \texttt{md} section is set to 1/3 of the available space; otherwise, it is set to 1/2 of the available space. For example, if the \texttt{.data} and \texttt{.bss} sections take 1 kilobytes in a RAM of 4 kilobytes, the space available is 3 kilobytes, so the size of the \texttt{md} is set to 1 kilobytes.

The \texttt{sp} section is used to store the address of the next available memory space of the SFS (similar to the stack pointer), called STP (\texttt{SnapShot Top Pointer}), because a stack pointer is needed for a stack data structure. To protect the STP from being affected by SEUs, the size of the \texttt{sp} section is set to 6 bytes and 3 STP duplicates are stored in this section. Because we assume that two or more flipped bits cannot concurrently exist in RAM, only one STP duplicate could be altered by the flipped bit. The altered STP is easily excluded by comparing the values of the three STP duplicates, yielding the correct STP value.


\subsection{Injected Code Segments}

\textbf{Terms: duplication vs. saving vs. copy}

We categorize the code injected into the target program into code segments based on their functions. Each segment performs a set of operations, handles a specific action, such as CRC calculation, and can be inject as a whole. Each segment is assigned with a unique ID used to refer to the segment during the code injection process. Below is a description of the code segments.

\begin{itemize}

\item The \texttt{CRC Calculation} segment (ID: \texttt{CC}) is used to calculate the CRC checksum of a given memory region, i.e., the stack frame. In our implementation, CRC-16 is used, which uses x registers to perform the calculation.

\item The \texttt{Stack Frame Copy} segment (ID: \texttt{FC}) is used to copy the stack frame to a given destination, and is used by both the stack frame saving and restoring processes.

\item The \texttt{STP Update} segment (ID: \texttt{SU}) is used to update the STP. First, it obtains the correct STP value by comparing the three STP duplicates. Then, all three STP duplicates are updated. The segment increases the STP to save a stack frame in the SFS, and decreases the STP to release a stack frame from the SFS.

\item The \texttt{Stack Frame Size Saving} segment (ID: \texttt{SS}) is used to save the size of stack frame size of the current function in the stack, discussed in Section xxx.

\end{itemize}

\begin{figure*}[t]
\centering
\includegraphics[width=1\textwidth]{figures/code_inject_process_v3}
%\includegraphics[width=\columnwidth]{figures/code_inject_process_v3}
\caption{Code injection Process}
\label{fig:code_injection_process}
\end{figure*}

\subsection{The ASM Handler}

The ASM Handler handles the code injection. It is designed to consist of three loose coupled modules: the \texttt{Reader}, the \texttt{Scanner}, and the \texttt{Injecter}, shown in Figure \ref{fig:code_injection_process}. A metadata is created to assist the code categorization and injection. Below is a description of the metadata and each module of the ASM Handler.

\subsubsection{ASM Metadata}

Each line of assembly code is associated with a metadata, which represents this line of code. The metadata categorizes the code into 3 categories, shown in Listing \ref{lst:metadata_example}. A \texttt{directive} is used to specify assembly code information, such as system architecture (line 1) and section (line 2), define label (line 3) ,label type (line 4), etc. A \texttt{label} is used to identify a location in the assembly code (line 5). In this example, label \texttt{main} specifies a location where the main function starts. A \texttt{instruction} is used to identify a instruction that will be executed by the microprocessor (line 6-8). The metadata also stores the code inject information, which specifies whether code is injected after this line, and the type of code to be injected.

\begin{lstlisting}[float=tb,label=lst:metadata_example,caption=Assembly Code Example]
.arch atmega644					% directive
	.text								% directive
.global	main						% directive
	.type	main, @function		% directive
main:									% label
	push r28							% instruction
	push r29							% instruction
	...
\end{lstlisting}

\subsubsection{Reader}

The Reader is used to read the assembly code file and generate a metadata list. It reads each line of the assembly code and generates a metadata node based on the assembly code. The metadata node is then appended to the metadata list. For example, the Reader generates a list with 7 nodes after it reads the assembly code in Listing \ref{lst:metadata_example}, shown in Figure \ref{fig:code_inject_process}.

\subsubsection{Scanner}

The Scanner is used to scan the metadata list, and mark the metadata nodes based on specified operations performed by the corresponding code. The marked metadata node indicates that code segments will be injected either before or after the corresponding line of code.

Since function calls follow the same process, by analyzing the execution sequence and the assembly code with the optimization level set to \texttt{none}, we identify the key operations where code segments should be injected. Below is a list of the key operations.

\begin{itemize}
\item The \texttt{Stack Frame Establishing} operation is used to 
\item The \texttt{Stack Frame Pointer Saving} operation is used to 
\item The \texttt{Function Return} operation is used to 
\end{itemize}

The Scanner scans each node in the metadata list, checks if the code represented by the node performs one of the key operations, and marks each node with a parameter set $\left\{S_{1}, S_{2}, ... S_{n}, P\right\}$, where $S_{1}$ -- $S_{n}$ are the IDs of the code segments to be injected and $P$ indicates the position of the injection (after or before the code). The node where no code will be injected is marked with an empty parameter set, $\left\{, \right\}$. For example, the metadata node that represents the xxx operation is marked with the parameter set xxx, indicating code segments xx, xx and xx to be injected before/after the represented line of code.
%The Scanner is used to scan the metadata list. Each metadata node is scanned and passed to the state machine by the Scanner. Based on the current state of the state machine after the metadata node is passed, the scanner either processes the next node, or marks the current node with parameters, indicating if code will be injected before or after the node, and what code will be injected.  For example, in Figure \ref{fig:code_inject_process}, two nodes are marked with parameters \texttt{C, B} and \texttt{M, A}, which respectively indicate CRC calculation and memory duplication code will be injected before and after corresponding nodes.

The Scanner also extracts two parameters from the metadata list. i) The Scanner detects if the \texttt{malloc} instruction is called in the target program, indicating whether the \texttt{heap} section in RAM is used, and determining the RAM section size used to store the stack frame duplicates, discussed in Section \ref{sec:memory_sections}. ii) The Scanner extracts the size of each function's stack frame by scanning the assembly code used to establish the stack frame, ``\texttt{sbiw r28, n}'', yielding a stack frame of size $n + 6$. The \texttt{n} bytes are used to store the arguments and local variables, and the additional $6$ bytes are used to store the return address, CRC, and the stack frame size, each of which takes 2 bytes.
%For each function's assembly code scanned, the Scanner also extracts two parameters. i) \texttt{Use\_Heap}, which is a boolean, is obtained by scanning if the \texttt{malloc} instruction is used in the target code. It indicates whether the \texttt{heap} section in RAM is used, determining the RAM section size used to store the stack frame duplicates, discussed in Section \ref{sec:memory_sections}. ii) \texttt{Stack\_Frame\_Size} is obtained by scanning the assembly code used to establish the stack frame, ``\texttt{sbiw r28, n}'', yielding a stack frame of size $n + 6$. The \texttt{n} bytes are used to store the arguments and local variables, and the additional $6$ bytes are used to store the return address, CRC, and the stack frame size, each of which takes 2 bytes.

\subsubsection{Injecter}

The Injecter is used to inject code segments into the target assembly code. It scans the metadata list again. When a node marked with non-empty parameter set is scanned, the Injecter injects the code segments specified in the parameter set in the position (before or after the corresponding code) specified by parameter $P$ in the parameter set. Finally, a modified assembly code file is generated, which will be assembled and linked to an executable file.


\subsection{Modified Function Execution Process}

Additional operations are added to the original function execution process by the injected code sections, performing CRC calculation, memory duplication, and other supporting operations such as register saving and restoring. Since the code sections are injected into the beginning and end of a function, the modified function execution process can be categorized into two categories, \texttt{Modified Function Invocation Process} and \texttt{Modified Function Return Process}. Below is a description of the modified function execution process.

\subsubsection{Modified Function Invocation Process}

The code segments injected into the beginning of a function is used to calculate CRC and save the duplicate of a given memory region, shown in Figure \ref{fig:modified_function_operation_pre_execution}. Figure \ref{fig:modified_function_operation_process_pre_execution} shows the execution process of the Pre-Execution Code; Figure \ref{fig:modified_function_operation_stack_pre_execution} shows the stack changes based on the Pre-Execution Code. In the execution process diagram, the white ovals show the operations performed by the original code, and the shaded ovals show the operations performed by the injected code. Each operation is labeled with a number. In the stack change diagram, \texttt{SP} is the stack pointer, and \texttt{Y} is the stack frame pointer. The numbers below each stack show the operations that changed the stack.

When function \texttt{B} is called by function \texttt{A}, the return address is pushed into the stack automatically by the function call instruction (step 1). To calculate CRC, multiple registers are used, so they must be saved before the CRC calculation process and restored when the process is finished. To avoid the calculated CRC saved in the registers being overwritten when the registers are restored, two bytes (zeros) are pushed into the stack as a placeholder (step 2) for the CRC result before the registers used to calculate CRC is saved (step 3). After the CRC of function \texttt{A}'s stack frame is calculated (step 4), the CRC result is saved to the placeholder location (step 5). The registers used to calculate CRC are then restored (step 6).

Next, the stack frame of the caller, function \texttt{A}, has to be saved. The registers used to save the stack frame are pushed into the stack (step 7). Then, the correct STP is selected by comparing the values of the three STP duplicates (step 8). Using the correct STP, the specified memory is then copied and saved in SFS (step 9). After the three STP duplicates are updated (step 10), the registers used are restored (step 11).

After the stack frame pointer of function \texttt{B} is saved (step 12) and the stack frame is established (step 13), the stack frame size of the callee, function \texttt{B}, is pushed into the stack (step 15), which is a key operation in the injected code. 

When a function is called, the return address is pushed into the stack which is used when the function returns. However, the callee function does not have context information about its caller, including the caller's stack frame address and size. It is thus impossible for the callee to calculate the CRC of the caller's stack frame and duplicate the stack frame without the information of the caller's stack frame size. To solve this problem, each function saves its stack frame size in the stack, which is used by its callee to perform CRC calculation and stack frame duplication.

\begin{figure}
        \centering
        \begin{subfigure}[b]{0.5\columnwidth}
                \includegraphics[width=\textwidth, height=12cm]{figures/modified_function_operations_process_pre_execution_v2}
                \caption{Process}
                \label{fig:modified_function_operation_process_pre_execution}
        \end{subfigure}~
        \begin{subfigure}[b]{0.5\columnwidth}
                \includegraphics[width=\textwidth, height=12cm]{figures/modified_function_operations_stack_pre_execution_v1}
                \caption{Stack}
                \label{fig:modified_function_operation_stack_pre_execution}
        \end{subfigure}
        \caption{Modified Function Invocation Process}\label{fig:modified_function_operation_pre_execution}
\end{figure}

\subsubsection{Modified Function Return Process}

The code segments injected into the end of a function is used to verify the stack frame of the caller function and restore the stack frame is an SEU is detected, shown in Figure \ref{fig:modified_function_operation_post_execution}. Figure \ref{fig:modified_function_operation_process_post_execution} shows the execution process of the Post-Execution Code; Figure \ref{fig:modified_function_operation_stack_post_execution} shows the system stack changes based on the Post-Execution Code. Again, in the execution process diagram, the white ovals show the operations performed by the original code, and the shaded ovals show the operations performed by the injected code. Each operation is labeled with a number. In the stack change diagram, \texttt{SP} is the stack pointer, and \texttt{Y} is the stack frame pointer. The numbers below each stack show the operations that changed the stack.

When function \texttt{B} returns, it first pops out its stack frame size from the stack (step 1). After the space used to store the arguments and local variables is released (step 2), the stack frame pointer is restored (step 3). The CRC of function \texttt{A}'s stack frame is then calculated and temporarily stored in two registers (step 4-6). Next, the calculated CRC is compared with the CRC saved in the stack (step 7). If the two CRCs do not match, the saved stack frame of \texttt{A} is restored to the stack and the STP is updated to release the space used to store the stack frame of \texttt{A} (step 8-12). Again, the stack frame size of function \texttt{A} saved in the stack is used in CRC compare and stack frame restoration. If the two CRCs match, the STP is updated (step 13-14). After the verification of \texttt{A}'s stack frame is completed, the CRC is popped out of the stack (step 15). Finally, function \texttt{B} returns, and the return address is popped automatically (step 16).


\begin{figure}
        \centering
        \begin{subfigure}[b]{0.5\columnwidth}
                \includegraphics[width=\textwidth, height=12cm]{figures/modified_function_operations_process_post_execution_v1}
                \caption{Process Post}
                \label{fig:modified_function_operation_process_post_execution}
        \end{subfigure}~
        \begin{subfigure}[b]{0.5\columnwidth}
                \includegraphics[width=\textwidth, height=12cm]{figures/modified_function_operations_stack_post_execution_v1}
                \caption{Stack Post}
                \label{fig:modified_function_operation_stack_post_execution}
        \end{subfigure}
        \caption{Modified Function Return Process}\label{fig:modified_function_operation_post_execution}
\end{figure}


\vspace{-15pt}
\section{Evaluation}\label{sec:evaluation}
\vspace{-10pt}
\vspace{-10pt}
\begin{figure}[h]
	\centering
	\includegraphics[width=0.75\textwidth]{figures/stacksize_usage_v3.pdf}
	\vspace{-5pt}
	\caption{Stack Usage of Test Applications}
	\label{fig:stacksize_usage}
\end{figure}
\vspace{-15pt}
Here we present our evaluation of the SEU protection approach. We first introduce three test applications with different degrees of stack dynamism. We then validate the correctness of our approach and analyze the relationship between protection efficacy and the SEU injection rate. Finally, we consider the overhead introduced by our approach, both in terms of space and execution speed. Ubuntu 13.10, with Linux kernel version 3.8, and GCC 4.1.2 are used.
\vspace{-15pt}
\subsection{Test Applications}
\vspace{-10pt}
To evaluate our approach under varying stack conditions and SEU injection rates, three AVR applications are considered. The stack usage pattern of each application is shown in Figure \ref{fig:stacksize_usage}. The x-axis represents execution time, and the y-axis represents stack size. Below is a description of each application.
\vspace{-5pt}
\begin{itemize}
\item The \textbf{Delay} application repeatedly executes a function that contains a delay of 2,040 clock cycles, implemented using a while loop, yielding low stack variability.
\item The \textbf{Double Function Calls} application repeatedly executes three functions --- function A calls B, and function B calls C --- yielding moderate stack variability.
\item The \textbf{Fibonacci} application repeatedly calculates the tenth Fibonacci number using recursion, yielding significant stack variability.
\end{itemize}
\vspace{-20pt}
\subsection{Validation}
\vspace{-10pt}
We first validate our approach and consider the SEU protection efficacy it affords. Recall the modified SRAM partition shown in Figure \ref{fig:modified_ram_map}. In our analysis, we ignore both the .data and the .bss sections, as well as the heap section. Data stored in the .data, .bss, and heap sections can be protected using well-known techniques based on cloning and comparison. We focus our analysis on stack frame protection. Again, the injected code segments used to protect the stack frames are designed to use only registers, and each segment requires only two bytes in the currently executing function's stack frame (for the return address).
%A embedded application is typically designed based on interrupts, and usually contains an infinite loop in a function, which never returns. If the function which has the infinite loop is not the \textit{main} function (although it is the \textit{main} function for most times), the stack frame of the function's caller is not protected. However, since the stack space containing the function and its callers' stack frames doesn't change during the execution of the applications, the space can be identified using static analysis and can be protected using cloning and comparison. This issue will be addressed in our future work.

We first assume that the currently executing function's frame, which includes the return address of the injected code segment, is not affected by SEUs. We use induction to prove the correctness of our approach. Suppose \textit{n} is the number of stack frames stored in the stack, excluding the frame for \textit{main}.

\textbf{Base Case:} If $n=0$, only the stack frame of \textit{main} is on the stack. When \textit{main} calls another function, say \textit{foo}, the stack frame for \textit{foo} is created. According to our assumption, the current stack frame (\textit{foo}'s) will not be affected by SEUs during execution. When \textit{foo} returns, the stack frame of \textit{main} is protected by our approach. So the stack frames of caller and callee are guaranteed to be correct if any function is invoked and returns when $n=0$.

\textbf{Induction Assumption:} Assume that the stack frames of callers and callees are guaranteed to be correct for $n=k$, where $k\geq 1$.

\textbf{Inductive Step:} Now consider $n=k+1$. Assume \textit{a} is the current function, which calls \textit{b}. According to our assumption, b's stack frame is not affected by SEUs. When \textit{b} returns, the stack frame of \textit{a} is protected by our approach. So the stack frames of callers and callees are guaranteed to be correct when $n=k+1$. The currently executing function's stack frame is assumed safe, and the stack frames of callers and callees are protected against SEUs during execution; by induction, the stack is guaranteed to be correct, assuming the current stack frame is never affected by SEUs.

To verify this claim, the AVR Simulator IDE~\cite{avrsimide} was used to manually inject SEUs, and to observe execution results. The results showed that each function is able to detect and fix SEUs introduced ``beneath'' the topmost stack frame.

However, if the stack frame of the current function is affected by an SEU, protection is not guaranteed. If the SEU changes key data, such as the return address or stack frame size, the current function will not execute as expected. We assume that only one SEU will occur during a given function execution, and that the SEU is uniformly likely to affect all bits in RAM. The probability of successful SEU protection can be expressed as:
\begin{equation}\label{eq_seu1}
p=1-\frac{c}{2s+e-c+6}
\end{equation}
Where $p$ is the probability of successful protection, $s$ is the stack size, $e$ is the size of the unused space in RAM, $6$ is the size of the three STP copies, and $c$ is the average size of a stack frame. Since the return address of the injected code segment is stored in the current stack frame, the two bytes for the return address are included in c. The total size of protected memory is $s+e+(s-c)+6$, where $s-c$ is the size of the stack frame copies stored in the \textit{md} section.

We extend our analysis to cases where more than one SEU may occur during a given function execution. Our approach succeeds when the following conditions are met: (i) the currently executing function's stack frame is not affected (so the return address of the injected code segment is not affected); (ii) at least two of the three copies of the caller's stack frame size are not affected; (iii) at least two of the three copies of the STPs are not affected; and (iv) at least one of the two caller's stack frames (the original and the backup copy saved in the \texttt{md} section) is not affected. To simplify the analysis, conditions (ii) and (iii) are strengthened, requiring that all three copies of the caller's stack frame size cannot be affected, and all three copies of the STPs cannot be affected. Since the strengthened conditions slightly reduce the probability of successful SEU protection (only 4 bytes are ignored), the real probability of protection is slightly higher than the presented results. The probability of successful SEU protection can be expressed as:
\begin{equation}\label{eq_seu2}
\begin{split}
&p=(1-\frac{c}{2s+e-c+6})^n*(1-\frac{6}{2s+e-2c+6})^n \\
&*(1-\frac{6}{2s+e-2c})^n*\{(1-\frac{2c}{2s+e-2c-6})^n \\
&+ \mathrm{C}_2^1*(1-\frac{c}{2s+e-2c-6})^n*[1-(1-\frac{c}{2s+e-3c-6})^n]\}
\end{split}
\end{equation}
Where $p$ is the probability of success, $s$ is the size of the stack, $e$ is the size of the unused space in RAM, $6$ is the size of the three stack frame size copies or the three STP copies, $c$ is the average size of the stack frame (including the return address of the injected code segment), and $n$ is the number of SEUs that occur during a function's execution. In equation \ref{eq_seu2}, $(1-\frac{c}{2s+e-c+6})^n$ is the probability that the currently executing function's stack frame is not affected by SEUs. $(1-\frac{6}{2s+e-2c+6})^n*(1-\frac{6}{2s+e-2c})^n$ is the probability that both copies of the caller's stack frame size and the three STPs are not affected by SEUs. Within the curly brackets, $(1-\frac{2c}{2s+e-2c-6})^n$ is the probability that both the original and the copy of the caller's stack frame are not affected by SEUs. $\mathrm{C}_2^1*(1-\frac{c}{2s+e-2c-6})^n*[1-(1-\frac{c}{2s+e-3c-6})^n]$ is the probability that either the original or the copy of the caller's stack frame is affected by SEUs. So $\{(1-\frac{2c}{2s+e-2c-6})^n + \mathrm{C}_2^1*(1-\frac{c}{2s+e-2c-6})^n*[1-(1-\frac{c}{2s+e-3c-6})^n]\}$ is the probability that at least one --- the original or the copy --- of the caller's stack frames is not affected.

In equation \ref{eq_seu2}, the number of SEUs that occur, $n$, can be expressed as:
\begin{equation}
n = \frac{y*l}{m}*f
\end{equation}
where $y$ is the number of clock cycles used to execute each instruction, $m$ is the frequency of the microprocessor, $l$ is the average number of function instructions, and $f$ is the SEU injection rate. Most AVR instructions require 2 clock cycles to execute, and the frequency of our ATmega644 is set to 10MHz.
\vspace{-15pt}
\begin{table}
	\center
    \begin{tabular}{|l|c|c|c|c|}
    \hline
    \textbf{Applications}   & \textbf{l} & \textbf{c} & \textbf{s} & \textbf{e}	\\ \hline
    Fibonacci             		& 42			& 10		& 60	   	& 2992		\\
\hline
    Double Function Calls       & 54			& 9			& 30        & 3022		\\ \hline
    Delay         				& 115			& 16		& 30		& 3022		\\
 \hline
    \end{tabular}
    \vspace{5pt}
    \caption {Application Characteristics}
    \label{tbl_application_parameters}
\end{table}
\vspace{-45pt}
\begin{figure}
\centering
\includegraphics[width=0.75\textwidth, height=130pt]{figures/success_probability_v2.pdf}
\vspace{-5pt}
\caption{SEU Protection Probability}
\label{fig:success_probability}
\end{figure}
\vspace{-20pt}
We now consider the relationship between SEU protection probability and SEU occurrence rate. To demonstrate the relationship, we collect the corresponding parameters for the three test applications using AVR Simulator IDE, as shown in Table \ref{tbl_application_parameters}. Figure \ref{fig:success_probability} plots the change in SEU protection probability as a function of SEU injection rate. The x-axis represents the rate at which SEUs are injected, and the y-axis represents the corresponding SEU protection probability. Each vertical line marks where the number of SEUs begins to exceed 1 (for each application). When only one SEU occurs during a given function execution (left side of the vertical line), the SEU protection probability is constant (Delay: 99.48\%, Double Function Calls: 99.71\%, Fibonacci: 99.68\%) because the only case the approach cannot handle is when the current frame is affected. When more than one SEU occurs during a given function execution (right side of the vertical line), the SEU protection probability increases because the SEUs may affect the stack frame of the current function, the stack frame sizes of the caller stored in the stack, the STPs, and stack frame copies stored in the \texttt{md} section. As the SEU occurrence rate increases, the SEU protection probability decreases, until it approaches 0. The lower the stack dynamism, the longer the function execution time, which increases the probability of SEU occurrence in the current stack frame. Low stack frame dynamism causes the SEU protection probability for Delay to drop significantly compared to the other applications.
\vspace{-15pt}
\subsection{Performance}
\vspace{-10pt}
Since the same code is injected for every function, the execution overhead is similar for all functions, varying only when an SEU is detected. Table \ref{tbl_speed_overhead} summarizes the overhead of each injected code segment. The second column lists the number of times each code segment executes (per function execution), the third column lists the number of instructions executed in each code segment, the fourth column lists the number of clock cycles spent executing each code segment, and the fifth column lists the ROM space overhead for each injected segment. $S$ denotes the size of the (recovered) stack frame. The \textit{CRC calculation} code segment and \textit{STP update} code segment execute twice for each function, and the \textit{frame copy} code segment executes either once or twice, depending on whether an SEU is detected. Each of the other code segments executes once for each function execution. Therefore, the minimum overhead introduced in terms of number of clock cycles is $62*S+304$, when an SEU is not detected. The worst case is $70*S+432$ clock cycles, when an SEU is detected. 
\vspace{-15pt}
\begin{table*}
	\center
    \begin{tabular}{|l|p{1.8cm}|p{2cm}|p{1.5cm}|p{1.5cm}|}
    \hline
   \textbf{Code Segment}   & \textbf{Number of Execution} & \textbf{Instructions} & \textbf{Clock Cycles} & \textbf{ROM Space}	\\ \hline
    CRC Calculation         & 2			& 24*S+1		& 27*S+4		& 50				\\ \hline
    CRC Save                & 1			& 13			& 26           	& 26				\\ \hline
    CRC Compare             & 1			& 27			& 52		   	& 64				\\ \hline
    Frame Copy				& 1 or 2	& 64+4*S		& 8*S+128      	& 50				\\ \hline
%    Frame Copy (Worst Case) & 1 		& 73+4*S      & 144+8*S      & 50				\\ \hline
    Frame Size save         & 1			& 18			& 34           	& 36				\\ \hline
    STP Initialization		& 1			& 16			& 28		   	& 32				\\ \hline
	STP Update				& 2			& 7				& 14			& 14				\\ \hline
	\textbf{Total (No Recovery)} & -	 	& 48*S+154     	& 62*S+304		& 272  		\\ \hline
	\textbf{Total (Recovery)}& -	 		& 52*S+218     	& 70*S+432		& 272 		\\ \hline
    \end{tabular}
	\vspace{5pt}
    \caption {Execution Overhead}
    \label{tbl_speed_overhead}
\end{table*}
\vspace{-25pt}
We next evaluate space overhead using the three test applications. The ROM space data was collected using \textit{avr-size}. The results are summarized in Figure \ref{fig:space_overhead}. The y-axis represents ROM size, in bytes. Delay and Fibonacci involve two functions, and Double Function Calls involves four. From Figure \ref{fig:space_overhead}, we can see that the ROM overhead for the Double Function Calls application is twice the Delay and Fibonacci applications. ROM overhead is related only to the number of functions in the program.
\begin{figure}[h]
\centering
\includegraphics[scale=0.47]{figures/space_overhead.pdf}
\caption{ROM Overhead}
\vspace{5pt}
\label{fig:space_overhead}
\end{figure}
\begin{figure}[h]
	\centering
	\includegraphics[width=0.75\textwidth]{figures/speed_overhead_line_chart_v1.pdf}
	\vspace{5pt}
	\caption{Execution Overhead}
	\label{fig:speed_overhead}
\end{figure}
\begin{figure}[h]
	\centering
	\includegraphics[width=0.75\textwidth]{figures/experiment1.pdf}
	\caption{SEU Experiment Results}
	\vspace{5pt}
	\label{fig:exp1_result}
\end{figure}
We next evaluate execution overhead. As shown in Table \ref{tbl_speed_overhead}, the execution overhead for every function call is determined by the size of the stack frame. We consider the execution overhead as a function of the average number of instructions executed between each call instruction (i.e., an inverse measure of call frequency). The execution overhead can be expressed as:
\begin{equation}\label{eq_seu1}
e=\frac{l+L}{l}
\end{equation}
Where \textit{L} is the number of injected machine instructions for each function, and \textit{l} is the average number of machine instructions executed between each call instruction. As summarized in Table \ref{tbl_speed_overhead}, $L=48*S+154$ when no SEUs are detected, and $L=52*S+218$ when an SEU is detected. The average frame size, $S$, is 20. The overhead results are summarized in Figure \ref{fig:speed_overhead}. The x-axis represents the average number of instructions executed between function invocations (\textit{l}), and the y-axis represents execution overhead, measured as the ratio between the execution speed of the original code and the modified code. The figure shows that given the same stack frame size, execution overhead is determined by stack dynamism. The less stack dynamism, the less speed overhead. The explanation is that within a given period of time, increased function calls lead to increased execution of the injected code. The results reveal an interesting tradeoff among dynamism, protection efficacy, and performance. Increasing dynamism offers better protection, but worse performance; decreasing dynamism offers better performance, but less protection. Knowledge of this tradeoff can be used to inform the function decomposition process, enabling embedded designers to appropriately balance protection efficacy and execution overhead.
\vspace{-12pt}
\subsection{Progressive Test}
\vspace{-10pt}
In verifying our approach on physical hardware, we emulate the occurrence of SEUs by flipping random bits in the target SRAM area. To perform auditable test runs, we developed an AVR application which continuously generates an increasing integer sequence, which is then sent to the UART interface at a controllable speed. A Python program running on a desktop is used to receive the sequence and observe the impact of flipped bits by monitoring the continuity of the sequence. A timer interrupt is used to trigger the occurrence of SEUs. The interrupt service routine function generates a random address within the range of the top of the stack and the end of RAM space, excluding the stack frame of the current interrupt, and then flips the bit at this location. The bit flip frequencies are set to $10^7$Hz, $1.25*10^6$Hz, $1.5625*10^5$Hz, $39062.5$Hz, and $9765.625$Hz.\footnote{These frequencies are derived from the built-in timer prescaler of the Atmega644, at 1, 8, 64, 256, and 1024, respectively}. An Atmega644 microprocessor is used in our experiments.

We declare (observable) failure when one of the following two situations occurs: (i) The AVR application stops generating integers; or (ii) the integer sequence received by the Python program becomes discontinuous. We monitor the integer sequence and record the maximum count before failure. The experimental results are summarized in Figure \ref{fig:exp1_result}. The x-axis represents the SEU injection frequency, and the y-axis represents the maximum count received by the Python program. The figure shows that as the SEU injection frequency increases, running time to failure decreases. This is explained as follows: As the SEU injection frequency increases, the probability that an SEU occurs in a critical area increases. When the frequency is extremely high (e.g. approximate 10 MHz), the program can hardly send any values. However, the observed SEU occurrence rate in outer space is approximate $10^{-6}SEU/bit$-$Day$~\cite{underwood1992observations}. Given the total RAM size of Atmega644 is 4K Bytes, the SEU occurrence rate of Atmega644 is 0.0032 SEU/day, which is significantly lower than the lowest frequency (9765.625 SEU/second) that we used. So this situation will be extremely rare in a realistic scenario.

\section{Conclusion}\label{sec:conclusion}

{\bf Future Work} External RAM; Memory duplicates compression; multidimensional parity checks.

\vspace{-15pt}
\section{Acknowledgments}\label{sec:acknowlegments}
\vspace{-10pt}
This work is supported by the National Science Foundation through
awards CNS-0745846 and CNS-1126344.
%
% The following two commands are all you need in the
% initial runs of your .tex file to
% produce the bibliography for the citations in your paper.
%\bibliographystyle{abbrv}
%\bibliography{sigproc}

% Use the following at camera-ready time to suppress page numbers.
% Comment it out when you first submit the paper for review.



%Now we get serious and fill in those references.  Remember you will have to run latex twice on the document in order to resolve those cite tags you met earlier.  This is where they get resolved. We've preserved some real ones in addition to the template-speak. After the bibliography you are DONE.
\vspace{-10pt}
{\footnotesize \bibliographystyle{acm}
\bibliography{./sigproc}}


%\theendnotes

\end{document}







