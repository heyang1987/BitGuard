\section{Introduction}\label{sec:introduction}

Humans have a longstanding curiosity about outer space. Since the launch of the first artificial satellite, Sputnik 1~\cite{sputnik1}, in 1957, over 6,000 satellites have been launched into space. There are more than 1,000 operating satellites in orbit around Earth~\cite{satellite:total} today, and an estimated 1,200 satellites will be launched over the next decade~\cite{satellite:next10years}. As of 2012, more than 130 manned spacecraft have been launched by the United States~\cite{space:shuttle:list}. There are currently two operational space stations, and seven more are planned over the next decade~\cite{space:station:tiangong2}\cite{space:station:almaz}\cite{space:station:opsek}\cite{space:station:tiangong3}. Given the high cost and vital importance of spacecraft rovers and satellites, as well as their increasing functionality and complexity, the hardware and software reliability requirements are stringent.

One of the most important factors that affect the reliability of spacecraft (and other equipment) is the quality of the constituent embedded components which control telemetry systems, command systems, attitude control systems, and more~\cite{fundamentals:space}. For example, the MSX (Midcourse Space Experiment) spacecraft, launched in the mid-1980s, was equipped with 54 embedded processors, running more than 275,000 lines of code, managing 19 subsystems~\cite{fundamentals:space}. Embedded software failures can cause serious consequences in this context. In 1996, the Ariane 5 spacecraft, which took 10 years and 7 billion (US) dollars to build, crashed due to the failure of the Flight Control Subsystem~\cite{ariane5}.

The environment outside the Earth's atmosphere is highly radioactive. The radiation is generated mainly by the sun and other stars and can cause damage to semiconductor devices~\cite{fundamentals:space}. One of the most common types of damage caused by radiation is the \textit{single event upset} (SEU). Extremely small electronic components (i.e., tens of nanometers~\cite{intel:chip:size}) are used in modern integrated circuitry; the components cannot carry much charge. As a result, one high-energy ionizing particle passing through an integrated circuit can release enough charge to change the state of a binary digit, causing a stored bit to change to its opposite value (i.e., a 0-bit can become a 1-bit, and vice-versa~\cite{fundamentals:space}). The damage caused by an SEU can range from system malfunction to system crash.

Modern approaches used to prevent and correct SEU errors often introduce additional hardware to the target system. In this paper, we present a {\em software-only} approach that detects and corrects SEUs in RAM. The paper focuses on the system stack, which is the most important and dynamic region in memory. The system stack is protected by injecting customized code into the target assembly generated by AVR-GCC. After injection, each callee computes and saves the checksum of its caller's current stack frame and duplicates the caller's stack frame when the callee enters its function body. Before the callee returns, it verifies the stack frame of the caller using the saved checksum and overwrites the stack frame using the duplicate, if an SEU is detected. Our approach changes the target system software and does not introduce additional hardware. Since our approach operates at the assembly level, it is language and application neutral. To demonstrate our approach, an AVR microprocessor, the ATmega644~\cite{atmel:avr}, is used in the paper. 

The main contributions of our work are as follows: (i) We present an approach that protects the system stack by injecting assembly code at the beginning and end of each application function. (ii) We present an implementation of the approach, using the popular AVR architecture as a target. (iii) We verify the protection efficacy of our approach and evaluate performance in terms of space and speed overhead using three applications with different stack usage patterns.

{\bf Paper Organization.} Section \ref{sec:related_work} summarizes key elements of related work. Section \ref{sec:background} provides background related to our approach, including the microprocessor architecture, AVR function call process, and AVR toolchain. Section \ref{sec:design} presents the design and implementation of our approach. Section \ref{sec:evaluation} presents an evaluation of the approach, with an emphasis on ROM size and execution speed overhead. Finally, Section \ref{sec:conclusion} concludes with a summary of contributions and pointers to future work.